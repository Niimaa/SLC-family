\documentclass[12pt]{report}
\usepackage[utf8]{inputenc}
\usepackage{graphicx}
\usepackage{multirow,longtable, subfiles, xcolor}
\usepackage{hyperref}
\graphicspath{{Images/}}
\title{SLC Super Family}
\author{Nima - Niklas}
\begin{document}
\maketitle
\textbf{Introduction}

The solute carrier family (SLC) contains group of membrane transporter protein which can be exchanger, cotransporter (both electrogenic and electroneutral) or passive transporter. The family includes around 400 members classified in 52 subfamilies. Most of the transporters of this super family are located in the cell membrane. 
The SLC series does not include members of transport protein families that have previously been classified by other widely accepted nomenclature systems including:
\begin{itemize}
    \item Primary active transports (e.g. ABC transporters)
    \item Ion channels (e.g. ENaC)
    \item Aquaporins (water channels)
\end{itemize}

In this study, the Bond graph approach is going to be used in order to model each member of each families in SLC group. Below is the table including all the 52 families along with their members. At the end of the table, an example of GLUT2 model based on Bond graph is added to briefly explain the modelling approach. GLUT2 (SLC2A2) is a member of SLC2 family which is a facilitated transporter.
Each model has a potential of being a physiome journal article, along with some information about each transport and their mechanism, and the validation of the model based on the experimental data if is existed in the literature otherwise needs to be asked a collaborator to provide them. 
\subfile{Table.tex}

\bibliography{References.bib}
\bibliographystyle{plain}
\end{document}